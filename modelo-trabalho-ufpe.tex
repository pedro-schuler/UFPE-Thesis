\documentclass{ufpe-article}

% Add your bibliography file
%%\addbibresource{references.bib}

\begin{document}

% Create the title page
\ufpetitle{Título do Artigo}
{Pedro Malta Schuler Caloête}
{DEPARTAMENTO DE CIÊNCIAS CONTÁBEIS E ATUARIAIS}
{BACHARELADO EM CIÊNCIAS CONTÁBEIS}
{DISCIPLINA}
{DATA}

% Generate table of contents
\ufpetoc

% Add abstract and keywords
\ufpeabstract{Este é o resumo do artigo acadêmico. Deve conter entre 150 e 250 palavras, apresentando de forma concisa os objetivos, metodologia, resultados e conclusões do trabalho. O texto deve ser em parágrafo único, justificado e com espaçamento simples.}
{palavra1; palavra2; palavra3; palavra4; palavra5}
{This is the abstract of the academic article. It should contain between 150 and 250 words, concisely presenting the objectives, methodology, results, and conclusions of the work. The text should be in a single paragraph, justified, and with single spacing.}
{keyword1; keyword2; keyword3; keyword4; keyword5}

% Start main content with proper page style
\ufpestartcontent

% Main content
\section{Introdução}
Este é um texto de exemplo para a introdução. Aqui você deve apresentar o tema do artigo, sua relevância, justificativa, objetivos e a estrutura do documento.

\subsection{Motivação}
A motivação para este trabalho está relacionada à necessidade de...

\subsection{Objetivos}
Os objetivos deste trabalho são:
\begin{itemize}
	\item Objetivo específico 1
	\item Objetivo específico 2
	\item Objetivo específico 3
\end{itemize}

\section{Fundamentação Teórica}
A fundamentação teórica apresenta os conceitos essenciais para a compreensão do trabalho.

\subsection{Conceito Principal}
O conceito principal deste trabalho é...

\section{Metodologia}
A metodologia utilizada neste trabalho foi...

\subsection{Coleta de Dados}
Os dados foram coletados através de...

\subsection{Análise dos Dados}
Para análise dos dados, foi utilizado o método...

\section{Resultados e Discussão}
\subsection{Resultados Principais}
Os principais resultados encontrados foram...

\begin{figure}[htb]
	\centering
	% \includegraphics[width=0.7\textwidth]{figura-exemplo.pdf}
	\caption{Exemplo de figura. Substitua com sua própria figura.}
	\label{fig:exemplo}
\end{figure}

\begin{table}[htb]
	\centering
	\caption{Exemplo de tabela.}
	\label{tab:exemplo}
	\begin{tabular}{cccc}
		\toprule
		\textbf{Coluna 1} & \textbf{Coluna 2} & \textbf{Coluna 3} & \textbf{Coluna 4} \\
		\midrule
		Valor 1           & Valor 2           & Valor 3           & Valor 4           \\
		Valor 5           & Valor 6           & Valor 7           & Valor 8           \\
		\bottomrule
	\end{tabular}
\end{table}

\subsection{Discussão}
A partir dos resultados obtidos, pode-se discutir que...

\section{Considerações Finais}
Com base nos resultados obtidos, conclui-se que...

\subsection{Trabalhos Futuros}
Para trabalhos futuros, sugere-se...

% Add references
%\ufpereferences

\end{document}
